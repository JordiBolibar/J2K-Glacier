\section{Berechnung von zus�tzlichen meteorologischen Gr��en}

\subsection{Absolute Luftfeuchte}
\subsubsection*{Sourcen:} 	org.unijena.regionWK.AP1.CalcAbsoluteHumidity.java
\subsubsection*{Beschreibung:}
Die absolute Luftfeuchte ist die vorhandene Masse an Wasserdampf in einer Volumeneinheit Luft. Die maximal m�gliche Masse an Wasserdampf, die pro Volumeneinheit Luft aufgenommen werden kann bis die Luft ges�ttigt ist, wird als S�ttigungsfeuchte bezeichnet. Sie wird mit dem von der Temperatur am jeweiligen Standort abh�ngigen S�ttigungsdampfdruck und dieser Temperatur nach folgender Formel berechnet:

\begin{equation}
%\label{potET}
maxHum = est \cdot \frac{216.7}{temp} \quad \mathrm{[gcm^{-3}]}
\end{equation}

\begin{tabular}{lllll}
mit & & &\\ 
& $est$ &  & S�ttigungsdampfdruck & $\mathrm{[kPa]}$\\
& $temp$ &  & Temperatur & $\mathrm{[K]}$.\\
\end{tabular}\\

Daraus wird die absolute Luftfeuchte unter Verwendung der gemessenen relativen Luftfeuchte ($rhum$) abgeleitet nach:
\begin{equation}
%\label{potET}
ahum = maxHum \cdot \frac{rhum}{100} \quad \mathrm{[gcm^{-3}]}
\end{equation}


%------------------------------------------------------------
\renewcommand{\refname}{\subsubsection*{Literatur}}
\begin{thebibliography}{}
\bibitem[\textsc{Brunotte} et al. (2001)\textsc{Brunotte} et al. ]{lexGeo}
\textsc{Brunotte,\,E., H.\,Gebhardt, M.\,Meurer, P.\,Meusburger \& J.\,Nipper} (Hrsg.) (2001 f.): Lexikon der Geographie in vier B�nden. Heidelberg: Spektrum.
\end{thebibliography} 
%------------------------------------------------------------

\subsection*{Eingabe:} 

\begin{tabular}{lllll}
%& & &\\ 
& $rhum$ &  & Relative Luftfeuchte & $\mathrm{[\%]}$\\
& $temperature$ & & Mittlere Lufttemperatur & $\mathrm{[�C]}$\\
\end{tabular}\\ 
    
\subsection*{Interne Variablen:}

\begin{tabular}{lllp{9.5cm}ll}
%& & &\\ 
& $rhumTemp$ &  & Temperatur f�r jede rhum-Station & $\mathrm{[�C]}$\\
& $absDist$ & & absoluter Abstand zwischen verwendeten Temperatur- und Relativen- Luftfeuchte-Messstation& $\mathrm{[m]}$ \\
& $est$ & & S�ttigungsdampfdruck & $\mathrm{[kPa]}$\\
& $maxHum$ & & S�ttigungsfeuchte & $\mathrm{[gcm^{-3}]}$\\
\end{tabular}\\


\subsection*{Ausgabe:}
\begin{tabular}{lllp{8.5cm}ll}
%& & &\\ 
& $ahum$ &  & Absolute Luftfeuchte & $\mathrm{[gcm^{-3}]}$\\
& $tempElevation$ & & Array von Temperatur-Stationsh�hen & $\mathrm{[m]}$ \\
& $tempXCoord$ & & Array von Temperatur-Stations-x-Koordinaten & $\mathrm{[-]}$ \\
& $tempYCoord$ & & Array von Temperatur-Stations-y-Koordinaten  & $\mathrm{[-]}$\\
& $rhumElevation$ & & Array von Relative-Luftfeuchte-Messstationsh�hen & $\mathrm{[m]}$ \\
& $rhumXCoord$ & & Array von Relative-Luftfeuchte-Messstations-x-Koordinaten & $\mathrm{[-]}$\\
& $rhumYCoord$ & & Array von Relative-Luftfeuchte-Messstations-y-Koordinaten & $\mathrm{[-]}$\\
& $regCoeffAhum$ & & Bestimmtheitsma� f�r Absolute-Luftfeuchte-Stationen & $\mathrm{[-]}$\\
\end{tabular}\\
 
%************************************************************ 

\newpage
\subsection{Globalstrahlung}

\subsubsection*{Sourcen:} 	org.unijena.regionWK.AP1.CalcDailySolarRadiation.java
\subsubsection*{Beschreibung:}
Die Globalstrahlung ($R_G$) wird aus der extraterrestrischen Strahlung ($R_0$) und der Bew�lkung errechnet. Der Bew�lkungsgrad wird hierbei aus dem Verh�ltnis der gemessenen Sonnenscheindauer ($S$) zur astronomisch m�glichen Sonnenscheindauer ($S_0$) bei unbedecktem Himmel unter Zuhilfenahme einer empirischen Beziehung nach der Formel von \r Angstr�m appoximiert. Die Globalstrahlung berechnet sich nach: 

\begin{equation}
R_G = R_0 \cdot (a + b \cdot \frac{S}{S_0}) \quad \mathrm{[Wm^{-2}]}
\end{equation}


%------------------------------------------------------------
\renewcommand{\refname}{\subsubsection*{Literatur}}
\begin{thebibliography}{}
\bibitem[\textsc{JAMSWiki} (2008)\textsc{JAMSWiki} ]{JAMSWiki}
\textsc{JAMSWiki} (2008): Hydrologisches Modell J2000, \\ $\left\langle http://jams.uni-jena.de/jamswiki/index.php/Hydrologisches\_Modell\_J2000 \right\rangle$ (Stand: 18.2.2008) (Zugriff: 14.12.2008).

\end{thebibliography} 
%------------------------------------------------------------

\subsection*{Eingabe:} 

\begin{tabular}{lllll}
%& & &\\ 
& $time$ &  & Zeitpunkt & \\
& $tempRes$ & & zeitliche Aufl�sung & $\mathrm{[d | h | m]}$ \\
& $sunh$ & & Sonnenscheindauer & $\mathrm{[h d^{-1}]}$\\
& $actSlAsCf$ & & Korrekturfaktor f�r Slope und Aspect & $\mathrm{[-]}$\\
& $latitude$ & & geographische Breite & $\mathrm{[�]}$\\
& $actExtRad$ & & extraterrestrische Strahlung & $\mathrm{[MJm^{-2}d^{-1}]}$\\
& $angstrom\_a$ & & \r Angstr�m Faktor a & $\mathrm{[-]}$\\
& $angstrom\_b$ & & \r Angstr�m Faktor b & $\mathrm{[-]}$\\
\end{tabular}\\ 


\subsection*{Interne Variablen:}

\begin{tabular}{lllll}
%& & &\\ 
& $declination$ &  & Deklination der Sonne& $\mathrm{[rad]}$\\
& $latRad$ & & geographische Breite& $\mathrm{[rad]}$ \\
& $ sunsetHourAngle$ & & Stundenwinkel zum Sonnenuntergang & $\mathrm{[rad]}$\\
\end{tabular}\\

\subsection*{Ausgabe:}
\begin{tabular}{lllll}
& $sunhmax$ & & maximale Sonnenscheindauer & $\mathrm{[h]}$\\
& $solRad$ & & t�gliche Globalstrahlung & $\mathrm{[MJm^{-2}d^{-1}]}$\\
\end{tabular}\\

%*****************************************************
\newpage
\subsection{Kurzwellige Ausstrahlung}

\subsubsection*{Sourcen:} 	org.unijena.regionWK.AP1.CalcShortWaveRadiation.java
\subsubsection*{Beschreibung:}
Die kurzwellige Ausstrahlung ($swR$) h�ngt von der Globalstrahlung ($R_G$) und dem R�ckstreuverm�gen einer Oberfl�che, dem Albedo ($alb$), ab. Sie berechnet sich nach:

\begin{equation}
swR = (1 - alb) \cdot R_G \quad \mathrm{[MJ m^{-2}]}
\end{equation}
%------------------------------------------------------------
%------------------------------------------------------------

\subsection*{Eingabe:} 

\begin{tabular}{lllll}
& $solRad$ & & Globalstrahlung & $\mathrm{[MJm^{-2}d^{-1}]}$\\
& $albedo$ & & Albedo & $\mathrm{[-]}$\\
\end{tabular}\\ 


\subsection*{Ausgabe:}
\begin{tabular}{lllll}
& $swRad$ & & kurzwellige Ausstrahlung & $\mathrm{[MJ m^{-2}]}$\\
\end{tabular}\\


%************************************************************
\newpage

\subsection{Langwellige Ausstrahlung}
\subsubsection*{Sourcen:} 	org.unijena.regionWK.AP1.CalcLongWaveRadiation.java
\subsubsection*{Beschreibung:}

Die langwellige Ausstrahlung der Erdoberfl�che und die atmosph�rische Gegenstrahlung werden gemeinsam als effektive langwellige Ausstrahlung ($RL$) berechnet. In die Berechnung gehen die Schwarzk�rperstrahlung nach Boltzmann, der Bew�lkungsgrad und eine empirische Funktion des Wasserdampfgehaltes der Luft ein: 

\begin{equation}
RL = B \cdot T^{4} \cdot (0.34 - 0.14 \cdot  \sqrt{ea}) \cdot (1.35 \cdot \frac{Rs}{Rs0} - 0.35) \quad \mathrm{[MJ m^{-2}]}
\end{equation}

\begin{tabular}{lllp{8.5cm}ll}
mit & & &\\ 
& $B$ &  & Stefan Boltzmann Konstante (=4.903E-9) & $\mathrm{[MJm^{-2}K^{-4}]}$\\
& $T$ & & absolue Lufttemperatur & $\mathrm{[K]}$\\
& $ea$ & & Dampfdruck & $\mathrm{[kPa]}$\\
& $Rs$ & & Globalstrahlung & $\mathrm{[MJm^{-2}d^{-1}]}$\\
& $Rs0$	&	& astronomisch m�gliche Solarstrahlung ohne Bew�lkung & $\mathrm{[MJm^{-2}d^{-1}]}$\\
\end{tabular}\\

%------------------------------------------------------------
\renewcommand{\refname}{\subsubsection*{Literatur}}
\begin{thebibliography}{}
\bibitem[\textsc{JAMSWiki} (2008)\textsc{JAMSWiki} ]{JAMSWiki}
\textsc{JAMSWiki} (2008): Hydrologisches Modell J2000, \\ $\left\langle http://jams.uni-jena.de/jamswiki/index.php/Hydrologisches\_Modell\_J2000 \right\rangle$ (Stand: 18.2.2008) (Zugriff: 14.12.2008).

\end{thebibliography} 
%------------------------------------------------------------

\subsection*{Eingabe:}
\begin{tabular}{lllll}
& $tmean$ & & Mittlere Lufttemperatur & $\mathrm{[�C]}$\\
& $rhum$ &  & Relative Luftfeuchte & $\mathrm{[\%]}$\\
& $extRad$ & & extraterrestrische Strahlung & $\mathrm{[MJm^{-2}d^{-1}]}$\\ 
& $solRad$ & & Globalstrahlung & $\mathrm{[MJm^{-2}d^{-1}]}$\\
& $elevation$ & & H�he & $\mathrm{[m]}$\\
\end{tabular}\\

\subsection*{Interne Variablen:}
\begin{tabular}{lllp{6.5cm}ll}
& $sat\_vapour\_pressure$ & & S�ttigungsdampfdruck & $\mathrm{[kPa]}$\\
& $act\_vapour\_pressure$ & & Dampfdruck & $\mathrm{[kPa]}$\\
& $clearSkyRad$ & & astronomisch m�gliche Solarstrahlung ohne Bew�lkung & $\mathrm{[MJm^{-2}d^{-1}]}$\\
& $BOLTZMANN$ & & Stefan Boltzmann Konstante & $\mathrm{[MJm^{-2}K^{-4}]}$\\
\end{tabular}\\

\subsection*{Ausgabe:}
\begin{tabular}{lllll}
& $lwRad$ & & langwellige Ausstrahlung & $\mathrm{[MJ m^{-2}]}$\\
\end{tabular}\\
%********************************************
\newpage

\subsection{Nettostrahlung}
\subsubsection*{Sourcen:} 	org.unijena.regionWK.AP1.CalcDailyNetRadiation.java
\subsubsection*{Beschreibung:}

Die Nettostrahlung ist die Differenz aus kurz- und langwelliger Ausstrahlung.  

%------------------------------------------------------------
\renewcommand{\refname}{\subsubsection*{Literatur}}
\begin{thebibliography}{}
\bibitem[\textsc{JAMSWiki} (2008)\textsc{JAMSWiki} ]{JAMSWiki}
\textsc{JAMSWiki} (2008): Hydrologisches Modell J2000, \\ $\left\langle http://jams.uni-jena.de/jamswiki/index.php/Hydrologisches\_Modell\_J2000 \right\rangle$ (Stand: 18.2.2008) (Zugriff: 14.12.2008).

\end{thebibliography} 
%------------------------------------------------------------

\subsection*{Eingabe:}
\begin{tabular}{lllll}
& $swRad$ & & kurzwellige Ausstrahlung & $\mathrm{[MJ m^{-2}]}$\\
& $lwRad$ & & langwellige Ausstrahlung & $\mathrm{[MJ m^{-2}]}$\\
\end{tabular}\\

\subsection*{Ausgabe:}
\begin{tabular}{lllll}
& $netRad$ & & Nettostrahlung & $\mathrm{[MJ m^{-2}]}$\\
\end{tabular}\\